%%%%%%%%%%%%%%%%%%%%%%%%%%%%%%%%%%%%%%%%%
% Programming/Coding Assignment
% LaTeX Template
%
% This template has been downloaded from:
% http://www.latextemplates.com
%
% Original author:
% Ted Pavlic (http://www.tedpavlic.com)
%
% Note:
% The \lipsum[#] commands throughout this template generate dummy text
% to fill the template out. These commands should all be removed when 
% writing assignment content.
%
% This template uses a Perl script as an example snippet of code, most other
% languages are also usable. Configure them in the "CODE INCLUSION 
% CONFIGURATION" section.
%
%%%%%%%%%%%%%%%%%%%%%%%%%%%%%%%%%%%%%%%%%

%----------------------------------------------------------------------------------------
%	PACKAGES AND OTHER DOCUMENT CONFIGURATIONS
%----------------------------------------------------------------------------------------

\documentclass{article}

\usepackage{fancyhdr} % Required for custom headers
\usepackage{lastpage} % Required to determine the last page for the footer
\usepackage{extramarks} % Required for headers and footers
\usepackage[usenames,dvipsnames]{color} % Required for custom colors
\usepackage{graphicx} % Required to insert images
\usepackage{listings} % Required for insertion of code
\usepackage{courier} % Required for the courier font
\usepackage{lipsum} % Used for inserting dummy 'Lorem ipsum' text into the template
\usepackage{setspace}
\usepackage{color}
\usepackage{comment}
\usepackage{caption}

\usepackage{hyperref}


\hypersetup{
    colorlinks=true,
    linkcolor=blue,
    filecolor=magenta,      
    urlcolor=cyan,
    breaklinks=true
}

%\usepackage[]{algorithm2e}
\usepackage{pdfpages}




%For python inclusion (http://widerin.org/blog/syntax-highlighting-for-python-scripts-in-latex-documents)
\definecolor{Code}{rgb}{0,0,0}
\definecolor{Decorators}{rgb}{0.5,0.5,0.5}
\definecolor{Numbers}{rgb}{0.5,0,0}
\definecolor{MatchingBrackets}{rgb}{0.25,0.5,0.5}
\definecolor{Keywords}{rgb}{0,0,1}
\definecolor{self}{rgb}{0,0,0}
\definecolor{Strings}{rgb}{0,0.63,0}
\definecolor{Comments}{rgb}{0,0.63,1}
\definecolor{Backquotes}{rgb}{0,0,0}
\definecolor{Classname}{rgb}{0,0,0}
\definecolor{FunctionName}{rgb}{0,0,0}
\definecolor{Operators}{rgb}{0,0,0}
\definecolor{Background}{rgb}{0.98,0.98,0.98}

% Margins
\topmargin=-0.45in
\evensidemargin=0in
\oddsidemargin=0in
\textwidth=6.5in
\textheight=9.0in
\headsep=0.25in

\linespread{1.1} % Line spacing

% Set up the header and footer
\pagestyle{fancy}
\lhead{\hmwkAuthorName} % Top left header
%\chead{\hmwkClass\ (\hmwkClassInstructor\ \hmwkClassTime): \hmwkTitle} % Top center head
\chead{\hmwkClass\ (\hmwkClassInstructor): \hmwkTitle} % Top center head
\rhead{\firstxmark} % Top right header
\lfoot{\lastxmark} % Bottom left footer
\cfoot{} % Bottom center footer
\rfoot{Page\ \thepage\ of\ \protect\pageref{LastPage}} % Bottom right footer
\renewcommand\headrulewidth{0.4pt} % Size of the header rule
\renewcommand\footrulewidth{0.4pt} % Size of the footer rule

\setlength\parindent{0pt} % Removes all indentation from paragraphs

%----------------------------------------------------------------------------------------
%	CODE INCLUSION CONFIGURATION
%----------------------------------------------------------------------------------------

\definecolor{MyDarkGreen}{rgb}{0.0,0.4,0.0} % This is the color used for comments
\lstloadlanguages{Perl} % Load Perl syntax for listings, for a list of other languages supported see: ftp://ftp.tex.ac.uk/tex-archive/macros/latex/contrib/listings/listings.pdf
\lstset{language=Perl, % Use Perl in this example
        frame=single, % Single frame around code
        basicstyle=\small\ttfamily, % Use small true type font
        keywordstyle=[1]\color{Blue}\bf, % Perl functions bold and blue
        keywordstyle=[2]\color{Purple}, % Perl function arguments purple
        keywordstyle=[3]\color{Blue}\underbar, % Custom functions underlined and blue
        identifierstyle=, % Nothing special about identifiers                                         
        commentstyle=\usefont{T1}{pcr}{m}{sl}\color{MyDarkGreen}\small, % Comments small dark green courier font
        stringstyle=\color{Purple}, % Strings are purple
        showstringspaces=false, % Don't put marks in string spaces
        tabsize=5, % 5 spaces per tab
        %
        % Put standard Perl functions not included in the default language here
        morekeywords={rand},
        %
        % Put Perl function parameters here
        morekeywords=[2]{on, off, interp},
        %
        % Put user defined functions here
        morekeywords=[3]{test},
       	%
        morecomment=[l][\color{Blue}]{...}, % Line continuation (...) like blue comment
        numbers=left, % Line numbers on left
        firstnumber=1, % Line numbers start with line 1
        numberstyle=\tiny\color{Blue}, % Line numbers are blue and small
        stepnumber=5 % Line numbers go in steps of 5
}

% Creates a new command to include a perl script, the first parameter is the filename of the script (without .pl), the second parameter is the caption
\newcommand{\perlscript}[2]{
\begin{itemize}
\item[]\lstinputlisting[caption=#2,label=#1]{#1.pl}
\end{itemize}
}


%----------------------------------------------------------------------------------------
%	DOCUMENT STRUCTURE COMMANDS
%	Skip this unless you know what you're doing
%----------------------------------------------------------------------------------------

% Header and footer for when a page split occurs within a problem environment
\newcommand{\enterProblemHeader}[1]{
\nobreak\extramarks{#1}{#1 continued on next page\ldots}\nobreak
\nobreak\extramarks{#1 (continued)}{#1 continued on next page\ldots}\nobreak
}

% Header and footer for when a page split occurs between problem environments
\newcommand{\exitProblemHeader}[1]{
\nobreak\extramarks{#1 (continued)}{#1 continued on next page\ldots}\nobreak
\nobreak\extramarks{#1}{}\nobreak
}

\setcounter{secnumdepth}{0} % Removes default section numbers
\newcounter{homeworkProblemCounter} % Creates a counter to keep track of the number of problems

\newcommand{\homeworkProblemName}{}
\newenvironment{homeworkProblem}[1][Problem \arabic{homeworkProblemCounter}]{ % Makes a new environment called homeworkProblem which takes 1 argument (custom name) but the default is "Problem #"
\stepcounter{homeworkProblemCounter} % Increase counter for number of problems
\renewcommand{\homeworkProblemName}{#1} % Assign \homeworkProblemName the name of the problem
\section{\homeworkProblemName} % Make a section in the document with the custom problem count
\enterProblemHeader{\homeworkProblemName} % Header and footer within the environment
}{
\exitProblemHeader{\homeworkProblemName} % Header and footer after the environment
}

\newcommand{\problemAnswer}[1]{ % Defines the problem answer command with the content as the only argument
\noindent\framebox[\columnwidth][c]{\begin{minipage}{0.98\columnwidth}#1\end{minipage}} % Makes the box around the problem answer and puts the content inside
}

\newcommand{\homeworkSectionName}{}
\newenvironment{homeworkSection}[1]{ % New environment for sections within homework problems, takes 1 argument - the name of the section
\renewcommand{\homeworkSectionName}{#1} % Assign \homeworkSectionName to the name of the section from the environment argument
\subsection{\homeworkSectionName} % Make a subsection with the custom name of the subsection
\enterProblemHeader{\homeworkProblemName\ [\homeworkSectionName]} % Header and footer within the environment
}{
\enterProblemHeader{\homeworkProblemName} % Header and footer after the environment
}

%----------------------------------------------------------------------------------------
%	NAME AND CLASS SECTION
%----------------------------------------------------------------------------------------

\newcommand{\hmwkTitle}{Assignment\ \#3 } % Assignment title
%\newcommand{\hmwkDueDate}{Monday,\ January\ 1,\ 2012} % Due date
\newcommand{\hmwkClass}{Introduction to Web Science/595} % Course/class
%\newcommand{\hmwkClassTime}{10:30am} % Class/lecture time
\newcommand{\hmwkClassInstructor}{Dr. Nelson} % Teacher/lecturer
\newcommand{\hmwkAuthorName}{Francis Pruter} % Your name

%----------------------------------------------------------------------------------------
%	TITLE PAGE
%----------------------------------------------------------------------------------------

\title{
\vspace{2in}
\textmd{\textbf{\hmwkClass:\ \hmwkTitle}}\\
%\normalsize\vspace{0.1in}\small{Due\ on\ \hmwkDueDate}\\
%\vspace{0.1in}\large{\textit{\hmwkClassInstructor\ \hmwkClassTime}}
\vspace{0.1in}\large{\textit{\hmwkClassInstructor}}
\vspace{3in}
}

\author{\textbf{\hmwkAuthorName}}
\date{Saturday, 4 October, 2014} % Insert date here if you want it to appear below your name

%----------------------------------------------------------------------------------------

\begin{document}

\maketitle



%----------------------------------------------------------------------------------------
%	TABLE OF CONTENTS
%----------------------------------------------------------------------------------------

%\setcounter{tocdepth}{1} % Uncomment this line if you don't want subsections listed in the ToC

\newpage
\tableofcontents
\newpage

%----------------------------------------------------------------------------------------
%	PROBLEM 1
%----------------------------------------------------------------------------------------

% To have just one problem per page, simply put a \clearpage after each problem

\begin{homeworkProblem}
\begin{verbatim}
1.  Download the 1000 URIs from assignment #2.  "curl", "wget", or
"lynx" are all good candidate programs to use.  We want just the
raw HTML, not the images, stylesheets, etc.

from the command line:

% curl http://www.cnn.com/ > www.cnn.com

% wget -O www.cnn.com http://www.cnn.com/

% lynx -source http://www.cnn.com/ > www.cnn.com

"www.cnn.com" is just an example output file name, keep in mind
that the shell will not like some of the characters that can occur
in URIs (e.g., "?", "&").  You might want to hash the URIs, like:

% echo -n "http://www.cs.odu.edu/show_features.shtml?72" | md5
41d5f125d13b4bb554e6e31b6b591eeb

("md5sum" on some machines; note the "-n" in echo -- this removes
the trailing newline.) 

Now use a tool to remove (most) of the HTML markup.  "lynx" will
do a fair job:

% lynx -dump -force_html www.cnn.com > www.cnn.com.processed

Keep both files for each URI (i.e., raw HTML and processed). 

If you're feeling ambitious, "boilerpipe" typically does a good
job for removing templates:

https://code.google.com/p/boilerpipe/\end{verbatim}


    \textbf{SOLUTION}
    This was solved using 2 bash scripts, one to download the HTML and another to 
remove the HTML using Lynx:
    \begin{enumerate}

    \lstinputlisting[breaklines=true, caption=Command: uniqueURL | ./dlURI2]{"dlURI2"}

    \item \textbf{Download HTML version of each URI:} The following bash script downloaded the HTML version of all 1000 URIs.


    \lstinputlisting[breaklines=true, caption=processURI]{"processURI"}

    \item \textbf{Remove the HTML from the downloaded websites:} This script goes 
through each of the downloaded websites and removes the HTML using lynx.
    \end{enumerate}




\end{homeworkProblem}
\clearpage

%----------------------------------------------------------------------------------------
%	PROBLEM 2
%----------------------------------------------------------------------------------------

\begin{homeworkProblem}

\begin{verbatim}
2.  Choose a query term (e.g., "shadow") that is not a stop word
(see week 4 slides) and not HTML markup from step 1 (e.g., "http")
that matches at least 10 documents (hint: use "grep" on the processed
files).  If the term is present in more than 10 documents, choose
any 10 from your list.  (If you do not end up with a list of 10
URIs, you've done something wrong).

As per the example in the week 4 slides, compute TFIDF values for
the term in each of the 10 documents and create a table with the
TF, IDF, and TFIDF values, as well as the corresponding URIs.  The
URIs will be ranked in decreasing order by TFIDF values.  For
example:

Table 1. 10 Hits for the term "shadow", ranked by TFIDF.

TFIDF	TF	IDF	URI
-----	--	---	---
0.150	0.014	10.680	http://foo.com/
0.085	0.008	10.680	http://bar.com/


You can use Google or Bing for the DF estimation.  To count the
number of words in the processed document (i.e., the deonminator
for TF), you can use "wc":

% wc -w www.cnn.com.processed
    2370 www.cnn.com.processed

It won't be completely accurate, but it will be probably be
consistently inaccurate across all files.  You can use more 
accurate methods if you'd like.  

Don't forget the log base 2 for IDF, and mind your significant
digits!
\end{verbatim}

\textbf{SOLUTION}

\begin{verbatim}
In order to solve this problem, I used a bash script and a python script:
    Below is the queryTerm script used and it asks the user for a query word.  
This script checks each processed website for a match (a minimum of 10 
websites must match).  It then calculates the number of matches per website 
as well as the total number of words in each website.  This will be used
 by the tfidfScore.py.

Additionally, it uses lynx to find the number of match that google will return.

The output (pageRank.dat) is in the github folder
\end{verbatim}


\lstinputlisting[breaklines=true, caption=queryTerm Script]{queryTerm}
%\lstinputlisting[breaklines=true, caption=pageRank.dat]{pageRank.dat}

\begin{verbatim}
    Below the tfidfScore.py will computer the TFIDF, TF, and IDF for each 
website with a match and prints out 10 of the matching URI.  
\end{verbatim}

\lstinputlisting[breaklines=true, caption=tfidfScore.py]{tfidfScore.py}
\lstinputlisting[breaklines=true, caption=tfidfScore.dat]{tfidfScore.dat}


\end{homeworkProblem}
\clearpage



%----------------------------------------------------------------------------------------
%   PROBLEM 3
%----------------------------------------------------------------------------------------

\begin{homeworkProblem}

\begin{verbatim}
3.  Now rank the same 10 URIs from question #2, but this time 
by their PageRank.  Use any of the free PR estimaters on the web,
such as:

http://www.prchecker.info/check_page_rank.php
http://www.seocentro.com/tools/search-engines/pagerank.html
http://www.checkpagerank.net/

If you use these tools, you'll have to do so by hand (they have
anti-bot captchas), but there is only 10.  Normalize the values
they give you to be from 0 to 1.0.  Use the same tool on all 10
(again, consistency is more important than accuracy).

Create a table similar to Table 1:

Table 2.  10 hits for the term "shadow", ranked by PageRank.

PageRank	URI
--------	---
0.9		http://bar.com/
0.5		http://foo.com/

Briefly compare and contrast the rankings produced in questions 2
and 3.

\end{verbatim}


\textbf{SOLUTION}
\begin{verbatim}
After trying multiple URI for page ranks, I mostly received N/A.  I ended 
up using the top-level domains for page ranks.  Utlizating R, I was able 
to normalize the output.

> google<-c(6,5,6,6,8,6,0,8,8,5)
> normalize <- function(x){(x-min(x))/(max(x)-min(x))}
> normalize(google)
 [1] 0.750 0.625 0.750 0.750 1.000 0.750 0.000 1.000 1.000 0.625



PageRank	URI
--------	---

1.000 		http://www.technologyreview.com/
1.000 		http://www.engadget.com/
1.000 		http://www.technologyreview.com/
0.750 		http://timehop.com/
0.750 		http://hardware.slashdot.org/
0.750 		http://appleinsider.com/
0.750 		http://blog.peerj.com/
0.625		https://photojojo.com/
0.625 		http://www.electronista.com/
0.000 		http://pressurenet.io/


Overall TFIDF and Google PageRank were quite different.  This 
is mostly do to the fact TFIDF uses a query word and ranks the site.  
Google PageRank ranks the overall site based on links pointing to you.

\end{verbatim}
\end{homeworkProblem}
\clearpage

%----------------------------------------------------------------------------------------
%   PROBLEM 4
%----------------------------------------------------------------------------------------

\begin{homeworkProblem}

\begin{verbatim}
4.  Compute the Kendall Tau_b score for both lists (use "b" because
there will likely be tie values in the rankings).  Report both the
Tau value and the "p" value.

See: 
http://stackoverflow.com/questions/2557863/measures-of-association-in-r
    -kendalls-tau-b-and-tau-c
http://en.wikipedia.org/wiki/Kendall_tau_rank_correlation_coefficient#Tau-b
http://en.wikipedia.org/wiki/Correlation_and_dependence

\end{verbatim}


\textbf{SOLUTION}
\begin{verbatim}
I used R to solve the Tau "b" and "p" value for TFIDF and Page ranks:
Tau "B": -0.1288848
P-value = 0.6831


> googleN
 [1] 0.750 0.625 0.750 0.750 1.000 0.750 0.000 1.000 1.000 0.625
> tfidf<-c(.631,.27,.243,.243,.118,.111,.09,.083,.083,.076)
> library(stats)
> cor.test(tfidf, googleN, method="kendall", alternative="greater")

	Kendall's rank correlation tau

data:  tfidf and googleN
z = -0.4765, p-value = 0.6831
alternative hypothesis: true tau is greater than 0
sample estimates:
       tau 
-0.1288848 

Warning message:
In cor.test.default(tfidf, googleN, method = "kendall", alternative = 
  "greater") :
  Cannot compute exact p-value with ties



\end{verbatim}
\end{homeworkProblem}
\clearpage

\begin{thebibliography}{9}

\bibitem{lamport94}\url{http://www.worldwidewebsize.com/}
\bibitem{lamport94}\url{http://icheckrank.com/multiple-pagerank-checker.php}
\bibitem{lamport94}\url{http://en.wikipedia.org/wiki/Kendall_tau_rank_correlation_coefficient#Tau-b}
\bibitem{lamport94}\url{http://stackoverflow.com/questions/5665599/range-standardization-0-to-1-in-r}


\end{thebibliography}


%----------------------------------------------------------------------------------------

\end{document}
